\chapter{Introduktion}

I det här dokumentet får du följa med på ett äventyr som jag påbörjade för några år sedan.
Jag hade fått inspirationen att börja ta min hälsa och mitt välmående mera på allvar.
Då var ju det självklara valet att börja nörda in mig på \enquote{hälsa och livsstil}.
Alla har vi ju hört på TV att \enquote{Det här knepet får dig att leva till 100 år!} eller läs på nyheterna att \enquote{A ökar risken för B}.
\newline\newline
Så jag började läsa på. Jag googlade fram och tillbaka efter svar på den bästa dieten och den bästa livsstilen som skulle få alla mina problem att försvinna.
Men problemet uppstod då all information jag hittade, verkade motsäga sig själv hundra gånger om:

\begin{itemize}
    \item \enquote{Ät X, det är bra för dina tarmar och matsmältningssystem\dots} ja okej, det låter ju lovande.
    \item \enquote{X ökar risken för hjärt- och kärlsjukdomar\dots} jaha, okej det låter ju inte så bra det då. Kanske jag bara skippar det helt och hållet?
    \item \enquote{Ny studie visar att Y minskar risken för cancer hos äldre\dots} oj, tack och lov. Så mina favoritsnacks kommer inte ge mig cancer då heller?
    \item \enquote{Forskare tycker att folk borde konsumera mindre av Y\dots} Jaha, okej. Nämen, forskare vet ju bäst. Dom är ju experter på sina områden. Jag vet ju inget själv, så bäst att lyssna på vad dom tycker.
    \item \enquote{Z innehåller \textit{näringsämne A}\dots men Z kan också orsaka \textit{sjukdom B}\dots} Okej så, det har hälsosamma näringsämnen, men jag kommer också få den där sjukdomen om jag äter det här?
\end{itemize}
\noindent
Hur ska man bestämma sig för vad man gör, när all information går kors och tvärs?
Eller när forskare tycker att man ska minska på sina favoritsnacks?
\newline\newline
Jag har alltid haft en väldig nyfikenhet för hur världen fungerar,
speciellt för den naturliga världen och den materiella omgivningen.
Det kan ju förklara varför jag fick ett hungrigt tycke för matematik, fysik och naturvetenskap i allmänhet, redan i ung ålder.
Mitt dilemma var ju att jag kunde inte fatta nån logik i det som jag läste.
Jag tyckte mig inte finna nån logisk förklaring till alla påståenden som fanns i informationsinlägg till höger och vänster.
\newline\newline
Jag hade inte en tillräcklig kunskapsgrund, på vilken jag kunde utvärdera informationen som jag läste.
Jag ville förstå den vetenskapliga grunden bakom hur allt detta fungerade.
Det har varit mitt äventyr de senaste åren, att utveckla mitt vetenskapliga kunnande och tänkade kring \enquote{hälsa och livsstil}.


\section{Motivering}

Det här dokumentet fungerar som en samling av idéer och teorier kring \enquote{hälsa och livsstil}.
Den huvudsakliga motiveringen är egentligen för min egen skull, att jag ska ha en plats att försöka sätta ner min egen kunskap på.
Dokumentet ska också vara ett konkret sätt att visa hur jag själv utvecklar mitt kritiska tänkande.


\section{Mål}

Jag har inget specifikt mål i åtanke.
Förutom att själv skriva ner mina tankar och idéer,
och kunna kritiskt utvärdera dom gentemot olika argument som existerar ut i världen.
\newline\newline
Om du som läsare mot förmodan hänger med på mitt äventyr,
och lär dig ett och annat längs med vägen,
så är det väl bra att jag kan bidra med sådant på köpet.

\clearpage