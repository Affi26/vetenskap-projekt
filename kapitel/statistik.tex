
\chapter{Statistik}


\section{Vad statistik är}


\section{Vad statistik inte är}


\section{Statistik som ett verktyg}


\section{Normalfördelningen}

Normalfördelningen, även kallad Gaussfördelningen, är en kontinuerlig sannolikhetsfördelning som kännetecknas av sin klockformade kurva.
Den beskrivs av två parametrar: medelvärdet $\mu$, som anger fördelningens centrum, och standardavvikelsen $\sigma$, som bestämmer spridningen.
Täthetsfunktionen för en normalfördelning ges av:
\begin{equation*}
    f(x) = \frac{1}{\sigma \sqrt{2\pi}} \exp\left( -\frac{(x - \mu)^2}{2\sigma^2} \right)
\end{equation*}
Normalfördelningen är viktig inom statistik och sannolikhetsteori eftersom många naturliga fenomen approximativt följer denna fördelning enligt centrala gränsvärdessatsen.
\begin{figure}[H]
    \centering
    \includegraphics[width=1\textwidth]{bilder/normal_fordelning_1.png}
    \caption{Normalfördelning med medelvärde $\mu$ och standardavvikelse $\sigma$.}
\end{figure}
\noindent
En trevlig egenskap hos normalfördelningen är att 95\% av alla datapunkter kommer att ligga inom ungefär 2 standardavvikelser från medelvärdet.
\begin{figure}[H]
    \centering
    \includegraphics[width=1\textwidth]{bilder/normal_fordelning_2.png}
    \caption{95\% av datat ligger inom ungefär 2 standardavvikelser från medelvärdet.}
\end{figure}


\section{Centrala gränsvärdessatsen}

Den korta sammanfattningen av centrala gränsvärdessatsen (GVS) är att stickprovsmedelvärdet är normalfördelat. 

\subsection{Empirisk demonstration}

Vanligtvis introduceras GVS genom en massa matematiska formaliteter,
vilka är bra att ha, men dom kan ibland skapa onödiga förvirringar.
Låt oss därför börja med en empirisk demonstration av GVS.
\newline\newline
\textbf{Exempel: kasta tärning}
\newline\newline
Vi kommer att generera ett slumpmässigt stickprov med 10 tärningskast, 
och beräkna medelvärdet och standardavvikelsen för stickprovet.
\newline\newline
Sedan kommer vi att upprepa proceduren och samla in fler stickprov med 10 tärningskast. 
Slutligen kommer vi att rita fördelningen av medelvärdena vi får från varje stickprov med hjälp av ett histogram. 
Vi kommer att se att fördelningen är ungefär normalfördelad.
\newline\newline
Vi antar en rättvis tärning med siffrorna 1--6, där varje sida har en sannolikhet på 1/6.
\begin{table}[H]
\centering
\begin{tabular}{c c}
\hline
Kast \# & Siffra \\
\hline
1 & 4 \\
2 & 2 \\
3 & 6 \\
4 & 3 \\
5 & 5 \\
6 & 1 \\
7 & 2 \\
8 & 4 \\
9 & 3 \\
10 & 6 \\
\hline
Medelvärde & 3.6 \\
Standardavvikelse & 1.68 \\
\hline
\end{tabular}
\caption{Exempel på ett stickprov av 10 tärningskast med medelvärde och standardavvikelse.}
\end{table}
\noindent
Låt oss säga att vi har utfört 10 stycken experiment, och fått följande resultat:
\begin{table}[H]
\centering
\begin{tabular}{cc}
\hline
Stickprov \# & Medelvärde \\ 
\hline
1  & 3.4 \\
2  & 3.7 \\
3  & 3.1 \\
4  & 3.9 \\
5  & 3.5 \\
6  & 3.2 \\
7  & 3.8 \\
8  & 3.6 \\
9  & 3.3 \\
10 & 3.7 \\
\hline
\end{tabular}
\caption{Stickprovsmedelvärden från 10 stickprov med $n=10$ tärningskast per stickprov.}
\end{table}
\noindent
Då kommer frekvensfördelningen för stickprovsmedelvärdet att se ut på följande sätt:
\begin{figure}[H]
    \centering
    \includegraphics[width=1\textwidth]{bilder/GVS_1.png}
    \caption{Frekvensfördelningen för stickprovsmedelvärdet från 10 stickprov, med $n=10$ tärningskast per stickprov.}
\end{figure}
\noindent
Fördelningen ser ju ganska jämn ut för tillfället.
Men låt oss nu säga att vi tar flera och flera stickprov.
Enligt gränsvärdessatsen kommer den här fördelningen att gå mot en normalfördelning då mängden stickprov ökar.
Det ser ut så här:
\begin{figure}[H]
    \centering
    \includegraphics[width=1\textwidth]{bilder/GVS_2.png}
    \caption{Frekvensfördelningen för stickprovsmedelvärdet då mängden stickprov ökar.}
\end{figure}
\noindent
Här uttrycks frekvensen relativt till mängden stickprov, för att också visualisera teoretiska normalfördelningarna bättre.
\newline\newline
Enligt våra empiriska stickprov så har stickprovsmedelvärdet störst sannolikhet att anta värdet 3.5,
vilket verkar stämma väl överens med det teoretiska väntevärdet för en 6-sidig tärning:
\begin{align*}
    \mathbb{E}[X] &= \sum_{k=1}^{6} k \cdot P(X=k) \\[0.3cm]
                  &= 1 \cdot \frac{1}{6} + 2 \cdot \frac{1}{6} + 3 \cdot \frac{1}{6} + 4 \cdot \frac{1}{6} + 5 \cdot \frac{1}{6} + 6 \cdot \frac{1}{6} \\[0.3cm]
                  &= \frac{1 + 2 + 3 + 4 + 5 + 6}{6} \\[0.3cm]
                  &= \frac{21}{6} \\[0.3cm]
                  &= 3.5 
\end{align*}

\subsection{Matematiskt teorem}

För att inte heller göra teorinördar besvikna, så kan vi förstås också presentera det officiella teoremet för GVS.
Det går ungefär nånting såhär:
\newline\newline
Centrala gränsvärdessatsen säger att för en följd av oberoende och identiskt fördelade slumpvariabler \( X_1, X_2, \ldots, X_n \) med ändligt medelvärde \(\mu = \mathbb{E}[X_i]\) och ändlig varians \(\sigma^2 = \mathrm{Var}(X_i)\) gäller att
\begin{equation*}
    S_n = \frac{1}{n} \sum_{i=1}^n X_i
\end{equation*}
är stickprovsmedelvärdet, och då konvergerar den normaliserade summan
\begin{equation*}
    Z_n = \frac{\sqrt{n}(S_n - \mu)}{\sigma}
\end{equation*}
i fördelning mot en standardnormalfördelning \( \mathcal{N}(0,1) \), det vill säga
\begin{equation*}
    Z_n \xrightarrow{d} \mathcal{N}(0,1) \quad \text{när} \quad n \to \infty.
\end{equation*}
Detta innebär att för stora \(n\) är fördelningen av \(S_n\) ungefär normalfördelad med medelvärde \(\mu\) och standardavvikelse \(\sigma / \sqrt{n}\).
\newline\newline
Notera även att omskriving av $Z_n$ kan göras så att:
\begin{equation*}
    Z_n = \frac{S_n - \mu}{\sigma / \sqrt{n}}
\end{equation*}
Detta är ju en standardiseringsmetod för en slumpvariabel $S_n$.
Standardisering av en slumpvariabel görs genom att subtrahera dess medelvärde, och dividera med dess standardavvikelse.
Det implicerar därav också att slumpvariabeln $S_n$ är normalfördelad:
\begin{equation*}
    S_n \sim \mathcal{N}(\mu, \frac{\sigma}{\sqrt{n}})
\end{equation*}
Stickprovsmedelvärdets standardavvikelse $\frac{\sigma}{\sqrt{n}}$ brukar också kallas för standardfel (eng. standard error, SE).
Stickprovsmedelvärdets fördelning kan således uppskattas med stickprovets egna medelvärde och standardavvikelse:
\begin{align*}
    \mu_{S_n} &\approx \frac{1}{n} \sum_{i=1}^n X_i \\[0.3cm]
    \sigma_{S_n} &\approx \frac{\sigma_X}{\sqrt{n}}
\end{align*}
Med andra ord, om vi har en mätning av ett normalfördelat värde, som ges av stickprovets medelvärde samt standardfel, \enquote{mean $\pm$ SE}, så kan vi räkna stickprovets standardavvikelse:
\begin{equation*}
    SD = SE \cdot \sqrt{n}
\end{equation*}
Därav kan vi uppskatta intervallet där 95\% av datapunkterna från ett stickprov faller:
\begin{equation*}
    95\% \text{ av datat} \approx [\mu - 2\sigma \ , \ \mu + 2\sigma]
\end{equation*}


\section{Statistisk signifikans}

Ordet \enquote{signifikant} (eng. significant) används överallt inom forskningslitteratur 
för att kommunicera att ett resultat anses vara tillräckligt pålitligt eller avvikande från slumpmässig variation för att förtjäna uppmärksamhet.
\newline\newline
Inom vardagligt språk uppfattas väl ordet \enquote{signifikant} som någonting med praktisk betydelse eller nytta.
Men den stora skillnaden är att ordet \enquote{signifikant} i kontexten av forskning faktiskt implicerar en betydelse som härstammar från statistik, och har inte egentligen nånting att göra med praktisk betydelse eller nytta.
\newline\newline
Ordet \enquote{signifikant} implicerar det som kallas statistisk signifikans.
Statistisk signifikans hör till området \enquote{hypotestestning}\dots
\newline\newline
Det här blir ganska snabbt tekniskt djupgående, eller hur?
Så låt oss se på ett exempel för att klargöra betydelsen.
\newline\newline
\textbf{Exempel: kroppslängd mellan två grupper av personer}
\newline\newline
Du har fått en uppfattning om att 12-åringar har samma kroppslängd som 18-åringar.
Du kan alltså formulera detta som en hypotes, betecknat $H_0$:
\begin{equation*}
    H_0: \text{Det är ingen skillnad på kroppslängden mellan 12-åringar och 18-åringar.}
\end{equation*}
Baserat på min erfarenhet av att observera 12-åringar och 18-åringar ute i världen så skulle jag tro att din hypotes är felaktig.
Min hypotes är att det finns en skillnad på kroppslängden mellan 12-åringar och 18-åringar.
\newline\newline
Vi kan testa dessa två hypoteser mot varandra:
\begin{align*}
    & H_0: \text{Det är ingen skillnad på kroppslängden mellan 12-åringar och 18-åringar.} \\[0.3cm]
    & H_1: \text{Det finns en skillnad på kroppslängden mellan 12-åringar och 18-åringar.}
\end{align*}
För att göra en matematisk formulering av hypoteserna, kan vi använda medellängderna $\mu_{12}$ och $\mu_{18}$ för respektive grupp:
\begin{align*}
    & H_0: \mu_{12} = \mu_{18} \\[0.3cm]
    & H_1: \mu_{12} \ne \mu_{18} \\[0.3cm]
\end{align*}
Vi kan också använda skillnaden $\delta = \mu_{12} - \mu_{18}$ som en egen variabel istället:
\begin{align*}
    & H_0: \delta = \mu_{12} - \mu_{18} = 0 \\[0.3cm]
    & H_1: \delta = \mu_{12} - \mu_{18} \ne 0 \\[0.3cm]
\end{align*}
Låt oss då göra en empirisk undersökning av detta.
Vi samlar 100 stycken 12-åringar, mäter deras längd, och beräknar ett medelvärde $\mu_{12}$ och en standardavvikelse $\sigma_{12}$.
Likadant gör vi för 100 stycken 18-åringar, där vi får ett medelvärde $\mu_{18}$ och en standardavvikelse $\sigma_{18}$.
\newline\newline
Låt oss säga att vi får följande frekvensfördelningar av längden för dom två stickproven med 12-åringar och 18-åringar:
\begin{figure}[H]
    \centering
    \includegraphics[width=0.95\textwidth]{bilder/signifikans_1.png}
    \caption{Frekvensfördelningen för stickprovet av längden hos 100 stycken 12-åringar, med $\mu_{12} = 151.0$ cm och $\sigma_{12} = 2.1$ cm.}
\end{figure}

\begin{figure}[H]
    \centering
    \includegraphics[width=0.95\textwidth]{bilder/signifikans_2.png}
    \caption{Frekvensfördelningen för stickprovet av längden hos 100 stycken 18-åringar, med $\mu_{18} = 169.9$ cm och $\sigma_{18} = 1.9$ cm.}
\end{figure}
\noindent
Om vi nu kombinerar båda fördelningarna och ser på fördelningen för skillnaden $\delta = \mu_{12} - \mu_{18}$ så ser det ut såhär:
\begin{figure}[H]
    \centering
    \includegraphics[width=0.95\textwidth]{bilder/signifikans_3.png}
    \caption{Frekvensfördelningen för skillnaden $\delta$ i längderna, med $\mu_\delta = -18.9$ cm och $\sigma_\delta = 2.9$ cm.}
\end{figure}
\noindent
Enligt den här grafen så skulle skillnaden vara negativ, $\delta = \mu_{12} - \mu_{18} < 0$, vilket ju betyder att skillnaden inte är noll.
Alltså att vår hypotes $H_1$ skulle vara sann, och $H_0$ inte stämmer.
\newline\newline
Men hur säkra kan vi vara på att $H_1$ är sann?
Om $H_0$ vore sann, alltså att $\delta = 0$, så kunde fördelningen för $\delta$ vara centrerad vid 0 istället:
\begin{figure}[H]
    \centering
    \includegraphics[width=0.95\textwidth]{bilder/signifikans_4.png}
    \caption{Frekvensfördelningen för skillnaden $\delta$ i längderna, då $\mu_\delta = 0$.}
\end{figure}
\noindent
Nu kommer vi till det kritiska läget där vi kan definiera statistisk signifikans.
Ifall vi antar att $H_0$ är sann, och $\delta$ borde vara centrerad kring 0,
så kan vi beräkna sannolikheten $p$ att observera vårt resultat, med skillnaden $\delta$ centrerad kring $-20$.
Om denna sannolikhet $p$ är mindre än en arbiträr signifikansnivå $\alpha$, så att $p < \alpha$, så säger vi att vårt observerade resultat är statistiskt signifikant.
\newline\newline
Om vi då väljer $\alpha$ att vara litet, och får $p < \alpha$, så kan vi säga att det är en väldigt liten chans ($p$) att vårt resultat producerades av rena slumpen,
så vi kan dra slutsatsen att observationen reflekterar en \enquote{riktig} skillnad eller effekt.
\newline\newline
Det är vanligt att välja $\alpha = 0.05$, alltså 5\%, men det är viktigt att komma ihåg att valet av $\alpha$ är totalt arbiträrt.
Valet av $\alpha$ har egentligen att göra med en mera generell filosofi kring hypotestestning och två fenomen som heter typ I och typ II fel.
Mera om det senare.
\newline\newline
Om vi nu då återgår till vår längdskillnad $\delta$.
Vi kan uppskatta våra fördelningar som normalfördelningar, 
den första som en $\mathcal{N}(-18.9, 2.9)$ fördelning och den andra som en $\mathcal{N}(0, 2.9)$ fördelning.
Då ser det ut såhär om vi visualiserar båda två i samma diagram:
\begin{figure}[H]
    \centering
    \includegraphics[width=0.95\textwidth]{bilder/signifikans_5.png}
    \caption{Normalfördelningarna av $\delta$, för båda hypoteserna $H_0$ och $H_1$.}
\end{figure}
\noindent
Nu antar vi att det är normalfördelningen för $H_0$ som gäller.
Alltså att $\mu_\delta = 0$.
Då beräknar vi sannolikheten att få ett värde $\delta = -18.9$ under täthetsfunktionen för $\mathcal{N}(0, 2.9)$:
\begin{align*}
    p
    &= P(\mu_\delta = -18.9 | \mu_\delta = 0) \\[0.3cm]
    &= \int_{-18.9 - 0.1}^{-18.9 + 0.1} f(x) \ dx \\[0.3cm]
    &= \int_{-18.9 - 0.1}^{-18.9 + 0.1} \ \frac{1}{2.9\sqrt{2\pi}} \exp\left( -\frac{(x - 0)^2}{2\cdot 2.9^2} \right) dx \\[0.3cm]
    &\approx 1.66 \cdot 10^{-11}
\end{align*}
Om vi väljer $\alpha = 0.05$, så ser vi att $p < \alpha$.
Så vi kan säga att vår observation för $\delta$ är statistiskt signifikant, och det är mycket osannolikt att vi skulle observera en skillnad på $\delta = -18.9$, ifall $\delta = 0$ faktiskt gäller.
Vår slutsats blir att $H_0$ sannolikt är fel, och att $H_1$ stämmer.
Det finns en statistiskt signifikant skillnad i längden mellan 12-åringar och 18-åringar.
\newline\newline
En sista detalj som är viktig att komma ihåg med statistisk signifikans och hypotestestning.
Vi beräknar inte sannolikheterna att hypoteserna är sanna, det vill säga $P(H_0 \text{ stämmer})$ och $P(H_1 \text{ stämmer})$.
Vi beräknar sannolikheten för den observation vi gör, ifall vi antar att $H_0$ stämmer, det vill säga $P(\text{observation } | H_0 \text{ stämmer})$.