\chapter{Biokemi \& Fysiologi}

Det här kapitlet ger inblick i några viktiga teoretiska koncept och definitioner inom biokemi och fysiologi.
Att ha en grundförståelse för dessa koncept gör det lättare att hänga med senare, när vi kommer att gå igenom och se på konkreta exempel ur forskningslitteratur.

\section{Energi \& massa}


\section{Reaktionshastighet}

Huruvida ett system befinner sig i kemisk jämvikt beror på om tidskonstanterna för de styrande kemiska reaktionerna är korta jämfört med de tidsskalor över vilka systemets tillstånd (temperatur och tryck) förändras.
De hastigheter med vilka kemiska reaktioner fortskrider beror på reaktanternas koncentration, temperaturen samt på om någon katalysator är närvarande.
Detta område kallas kemisk kinetik, och några av dess grundläggande samband kommer nu att översiktligt behandlas \cite{heywood_combustion_1988}.
\newline\newline
Många kemiska reaktioner av intresse inom biokemi kan beskrivas som binära reaktioner, där två reaktantmolekyler, M$_a$ och M$_b$, som har förmåga att reagera med varandra, kolliderar och bildar två produktmolekyler, M$_c$ och M$_d$:
\[
\ce{M_a + M_b <=> M_c + M_d}
\]
Lagen om massans verkan säger att den hastighet med vilken produkter bildas och den hastighet med vilken reaktanter förbrukas är proportionell mot produkten av koncentrationerna hos reaktanterna.
Koncentrationen av varje kemisk art upphöjs till potensen av dess stökiometriska koefficient $v_i$:
\[
R = k \prod_i [M_i]^{v_i}
\]
För reaktionen ovan ges reaktionshastigheten $R^+$ i den positiva riktningen (+), från reaktanter till produkter, av:
\[
R^+ = -\frac{d[M_a]^+}{dt} = \frac{d[M_c]^+}{dt} = k^+[M_a][M_b]
\]
Om reaktionen även kan fortskrida i omvänd riktning (-), ges den negativa reaktionshastigheten $R^-$ av:

\[
R^- = -\frac{d[M_c]^-}{dt} = \frac{d[M_a]^-}{dt} = k^-[M_c][M_d]
\]
Storheterna $k^+$ och $k^-$ är hastighetskonstanterna i positiv respektive negativ riktning för denna reaktion.
\textit{Nettohastigheten} med vilken produkter bildas eller reaktanter förbrukas ges därför av:
\[
R^+ - R^- =
\frac{d[M_c]^+}{dt} - \frac{d[M_c]^-}{dt}
=
-\frac{d[M_a]^+}{dt} - \frac{d[M_a]^-}{dt}
=
k^+[M_a][M_b] - k^-[M_c][M_d]
\]
Hastighetskonstanterna $k$ följer vanligtvis Arrhenius form:
\[
k = A \exp\left(-\frac{E_A}{RT}\right)
\]
Här kallas $A$ frekvensfaktorn eller den preexponentiella faktorn och kan vara en (måttlig) funktion av temperaturen.
Storheten $E_A$ är aktiveringsenergin.
Boltzmannfaktorn $\exp(-E_A/RT)$ anger den andel av alla kollisioner som har en energi större än $E_A$, det vill säga tillräcklig energi för att reaktionen ska kunna äga rum.
Det funktionella beroendet för $k$ på temperaturen $T$, samt konstanterna i Arrhenius-uttrycket, bestäms experimentellt.


\section{Proteiner \& enzymer}

\textbf{Proteiner} är stora biologiska molekyler uppbyggda av långa kedjor av aminosyror.
De är en av de viktigaste byggstenarna i alla levande organismer.
Proteinernas specifika funktion bestäms av deras tredimensionella struktur, som i sin tur beror på aminosyrasekvensen.
Proteiner har många olika roller:
\begin{itemize}
    \item strukturella komponenter (t.ex. kollagen)
    \item transport av molekyler (t.ex. hemoglobin)
    \item signalöverföring (t.ex. receptorer)
    \item reglering av genuttryck
    \item katalys av kemiska reaktioner
\end{itemize}
\textbf{Enzymer} är en särskild typ av proteiner som fungerar som \emph{biologiska katalysatorer}.
De ökar hastigheten på kemiska reaktioner utan att själva förbrukas.
Enzymer gör detta genom att sänka reaktionens aktiveringsenergi, vilket gör att reaktionen kan ske snabbare vid biologiska temperaturer.
\newline\newline
Enzymer är oftast mycket specifika.
Det innebär att ett enzym vanligtvis bara katalyserar en viss reaktion eller binder ett specifikt substrat.
Denna specificitet är avgörande för att cellens biokemiska processer ska kunna regleras noggrant.

\section{Fosforylering}

Fosforylering är fästandet av en fosfatgrupp till en annan molekyl, och är en grundläggande reaktion som krävs för att många biokemiska processer ska fortskrida:
\[
M + P_i \rightarrow MP
\]
Fosfatgruppen kan härstamma från olika håll, men den vanligaste källan är en fosfatgrupp som har donerats från en ATP molekyl:
\[
M + ATP \rightarrow MP + ADP
\]
Beteckningen P$_i$ syftar mera exakt på en inorganisk fosfatgrupp.
Det är en fosfatgrupp som har beteckningen PO$_4^{3-}$.


\section{ATP-syntes}

Cellens huvudsakliga \enquote{energi} tar formen av en fysisk molekyl som kallas ATP, adenosin trifosfat (eng. adenosine triphosphate).
Alla har vi ju hört att mitokondrien är cellens kraftverk,
vilket stämmer mycket riktigt eftersom mitokondrien ansvarar för den huvudsakliga syntetiseringen av ATP.
\begin{figure}[H]
    \centering
    \includegraphics[width=0.7\textwidth]{bilder/ATP_structure.png}
    \caption{ATP-molekylens struktur.}
\end{figure}
\noindent
Namnet ATP syftar på en adenosinmolekyl som har tre stycken inorganiska fosfatgrupper, P$_i$, fastsatta i sig.
Adenosinmolekylen kan maximalt ha tre fosfatgrupper kopplade på grund av de elektriska laddningarna och den energi som krävs för att bilda och bryta dessa bindningar.  
Dessa fosfatgrupper är högt energirika, och det är frigörandet av energi vid brytning av bindningar mellan dem som driver många biologiska processer.
\newline\newline
ATP:s huvudsakliga roll är att donera fosfatgrupper till olika reaktioner.
Om ATP donerar en fosfatgrupp så kallas den återstående molekylen för ADP, adenosin difosfat.
Om ADP ytterligare donerar en fosfatgrupp så återstår AMP, adenosin monofosfat.
Återigen kan AMP och ADP acceptera fosfatgrupper från olika reaktioner.
Dom olika stadierna av ATP, ADP och AMP agerar som signaler för cellens energitillstånd.
\newline\newline
ATP syntetiseras genom en kombination av en ADP molekyl och en P$_i$ molekyl:
\[
ADP + P_i \rightarrow ATP
\]
På grund av den negativa elektriska laddningen hos fosfatgrupperna finns det en elektrostatisk repulsion mellan ADP och P$_i$.
För att reaktionen ska kunna ske krävs därför en specifik kemisk miljö där molekylerna hålls i rätt orientering och där reaktionens övergångstillstånd stabiliseras.
\newline\newline
Detta sker med hjälp av ett enzym i mitokondrien som kallas \textbf{ATP-syntas}.
ATP-syntasen är en roterande proteinmotor.
När väteprotoner (H$^+$) strömmar genom en kanal i ATP-syntasen roterar en central del av enzymet, vilket orsakar konformationsförändringar i de katalytiska sätena.
\newline\newline
I dessa konformationslägen binds ADP och P$_i$ tätt i det katalytiska sätet, vilket skapar förhållanden där ATP kan bildas spontant.
Ytterligare rotation leder till att ATP släpps fri från enzymet.
För djupare läsning om ATP-syntasens struktur och funktion, se referenserna \cite{walker1998atp, leyva2003understanding,von2009essentials,lai2023structure}.
\newline\newline
Väteprotonerna som driver rotationen pumpas över mitokondriens inre membran av elektrontransportkedjan, där elektroner transporteras genom en serie proteinkomplex.


\section{Elektrontransportkedjan}

Elektrontransportkedjan förkortas ETC (eng. electron transport chain).
\begin{figure}[H]
    \centering
    \includegraphics[width=0.1\textwidth]{bilder/placeholder.png}
    \caption{\textcolor{red}{\textit{BILD HÄR}}}
\end{figure}


\section{TCA-cykeln}

Elektronerna (e$^-$) och väteprotonerna (H$^+$) som levereras till ETC härstammar från TCA-cykeln.
\begin{figure}[H]
    \centering
    \includegraphics[width=1\textwidth]{bilder/TCA-cykel.png}
    \caption{TCA-cykeln. Enzymer är markerade i röd text. Elektronleverantörerna NADH och FADH$_2$ är markerade i grön text.}
\end{figure}

\begin{figure}[H]
    \centering
    \includegraphics[width=0.5\textwidth]{bilder/TCA_1.png}
    \caption{Substrat som passerar till TCA-cykeln.}
\end{figure}


\section{Randle-cykeln}

Läs Hue et al. \cite{hue2009randle} för en sammanfattning.
\newline\newline
Långkedjiga fettsyror förkortas LCFA.
Triglycerider förkortas TAG.
\begin{figure}[H]
    \centering
    \includegraphics[width=1\textwidth]{bilder/randle_1.png}
    \caption{Glukos-fettsyra-cykeln (Randle-cykeln) beskriver en ömsesidig inhibering mellan glukosoxidering och fettsyraoxidering, främst i skelettmuskel och hjärta. Bild återskapad från Hue et al. \cite{hue2009randle}.}
\end{figure}

\begin{figure}[H]
    \centering
    \includegraphics[width=1\textwidth]{bilder/randle_2.png}
    \caption{Beta-oxidering inhiberar glykolys. Bild återskapad från Hue et al. \cite{hue2009randle}.}
\end{figure}

\begin{figure}[H]
    \centering
    \includegraphics[width=1\textwidth]{bilder/randle_3.png}
    \caption{Glykolys inhiberar beta-oxidering. Bild återskapad från Hue et al. \cite{hue2009randle}.}
\end{figure}